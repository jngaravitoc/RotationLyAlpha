\documentclass[usenatbib]{mn2e} 
\usepackage{amsmath} 
\usepackage{amssymb} 
\usepackage{graphics}
\usepackage{graphicx}
\usepackage{epsfig} 
\usepackage{float} 
\def\be{\begin{equation}}
\def\ee{\end{equation}}
\def\ba{\begin{eqnarray}}
\def\ea{\end{eqnarray}}

% To highlight comments 
\usepackage{color}
\definecolor{red}{rgb}{1,0.0,0.0}
\newcommand{\red}{\color{red}}
\definecolor{blue}{rgb}{0.1,0.3,0.9}
\newcommand{\blue}{\color{blue}}

\usepackage[normalem]{ulem}
\definecolor{darkgreen}{rgb}{0.0,0.5,0.0}
\newcommand{\SRK}[1]{\textcolor{darkgreen}{\bf SRK: \textit{#1}}}
\newcommand{\SRKED}[1]{\textcolor{darkgreen}{\bf #1}}

\newcommand{\LCDM}{$\Lambda$CDM~}
\newcommand{\beq}{\begin{eqnarray}}  
\newcommand{\eeq}{\end{eqnarray}}  
\newcommand{\zz}{$z\sim 3$} 
\newcommand{\apj}{ApJ}  
\newcommand{\apjs}{ApJS}  
\newcommand{\apjl}{ApJL}  
\newcommand{\aj}{AJ}  
\newcommand{\mnras}{MNRAS}  
\newcommand{\mnrassub}{MNRAS accepted}  
\newcommand{\aap}{A\&A}  
\newcommand{\aaps}{A\&AS}  
\newcommand{\araa}{ARA\&A}  
\newcommand{\nat}{Nature}  
\newcommand{\physrep}{PhR}
\newcommand{\pasp}{PASP}    
\newcommand{\pasj}{PASJ}    
\newcommand{\avg}[1]{\langle{#1}\rangle}  
\newcommand{\ly}{{\ifmmode{{\rm Ly}\alpha}\else{Ly$\alpha$}\fi}}
\newcommand{\hMpc}{{\ifmmode{h^{-1}{\rm Mpc}}\else{$h^{-1}$Mpc }\fi}}  
\newcommand{\hGpc}{{\ifmmode{h^{-1}{\rm Gpc}}\else{$h^{-1}$Gpc }\fi}}  
\newcommand{\hmpc}{{\ifmmode{h^{-1}{\rm Mpc}}\else{$h^{-1}$Mpc }\fi}}  
\newcommand{\hkpc}{{\ifmmode{h^{-1}{\rm kpc}}\else{$h^{-1}$kpc }\fi}}  
\newcommand{\hMsun}{{\ifmmode{h^{-1}{\rm {M_{\odot}}}}\else{$h^{-1}{\rm{M_{\odot}}}$}\fi}}  
\newcommand{\hmsun}{{\ifmmode{h^{-1}{\rm {M_{\odot}}}}\else{$h^{-1}{\rm{M_{\odot}}}$}\fi}}  
\newcommand{\Msun}{{\ifmmode{{\rm {M_{\odot}}}}\else{${\rm{M_{\odot}}}$}\fi}}  
\newcommand{\msun}{{\ifmmode{{\rm {M_{\odot}}}}\else{${\rm{M_{\odot}}}$}\fi}}  
\newcommand{\lya}{{Lyman$\alpha$~}}
\newcommand{\clara}{{\texttt{CLARA}}~}
\newcommand{\rand}{{\ifmmode{{\mathcal{R}}}\else{${\mathcal{R}}$ }\fi}}  
\newcommand{\hs}{{\hspace{1mm}}}  
\newcommand{\kms}{\,km~s$^{-1}$}

% definition to produce a "less than or similar to" symbol
\def\lsim{~\rlap{$<$}{\lower 1.0ex\hbox{$\sim$}}}

% definition to produce a "greater than or similar to" symbol
\def\gsim{~\rlap{$>$}{\lower 1.0ex\hbox{$\sim$}}}

\begin{document}

\title[Rotation in the Lyman-$\alpha$ line]{Effects of rotation on the
  Lyman-$\alpha$ line morphology in distant galaxies}
\author[N. Garavito and J.E. Forero-Romero]{
\parbox[t]{\textwidth}{\raggedright 
  Nicolas Garavito-Camargo.$^{1}$ 
  Jaime E. Forero-Romero$^{2}$ 
}
\vspace*{6pt}\\
$^{1}$Uni A
$^{2}$Uni B
}
\maketitle

\begin{abstract}

\end{abstract}
\begin{keywords}
galaxies: high-redshift - galaxies: star formation - line: formation
\end{keywords}


\section{Introduction}
\label{sec:intro}

Due to the resonant nature of the lyman alpha line, gas kinematics
play an important role shaping its morphology. In the literature there
has been extensive studies of outflow/inflow configurations. 

In this paper we study for the first time the impact of rotation on
the morphology of the Lyman $\alpha$ line. To isolate the effects of
rotation we focus on a simple system: the gas distribution is
spherical, with homogenous density and the gas rotates as a solid
body.

This paper is paper is structured as follows.



\section{Implementation of Bulk Gas Rotation}
\label{sec:implementation}

We implement into CLARA the simplest model whereby a sphere rotates
with homogeneous angular velocity. We define a cartesian coordinate
system with its origin at the center of the sphere and the rotation
axis to be the $z$-axis, the components in the bulk velocity field, $\vec{v}
= v_{x}\hat{i} + v_{y}\hat{j} + v_{z}\hat{k}$. in the gas can be written as 
 
\begin{subequations}
\begin{align}
    v_{x}=-\dfrac{y}{R}V_{\rm max}, \label{subeq1}\\
    v_{y}=\dfrac{x}{R}V_{\rm max}, \label{subeq2}\\
    v_{z}=0, \label{subeq3}
\end{align}
\end{subequations}

where $R$ is the radius of the sphere and $V_{\rm max}$ is the linear
velocity at the sphere's surface. The minus sign in the x-component of
the velocity indicates the direction of rotation, in this case we
assume that the angular velocity vector goes in the $\hat{k}$
direction.  The linear dependence of the velocity on the radial
distance describes the case of constant angular velocity
$\omega=V_{\rm max}/R$.  

We take the polar angle $\theta$ that a unitary vector makes with the
rotation axis as defined by the dot product $\cos\theta =
{\hat{u}\cdot\hat{k}}$. In the Section \ref{sec:results} we will
present in detail how the line differs at different observing angles
$\theta$. 



\section{Grid of Simulated Models}
\label{sec:models}

We compute the emergent Lyman-$\alpha$ line for several models with
differents values for the maximal rotational velocity, hydrogend optical
depth, dust optical depth and initial distributions of the photons
with respect to the gas. There are in total 60 models with the input
parameters summarized in Table  \ref{table:models}. 

\begin{table}
\begin{center}
\begin{tabular}{ccc}\hline
Velocity (\kms) & $V_{\rm max}$&$0,\ 50,\ 100,\ 200,\ 300$\\
Hydrogen Optical Depth & $\tau_{H} $ & $10^{5},\ 10^{6},\ 10^{7}$\\
Dust Optical Depth & $\tau_{A}$ & XXX \\
Photons Distributions & & Central, Homogeneous\\
\hline
\end{tabular}
\caption{
Values for the varying input parameters in CLARA. Taking into account
all the possible combinations for these models
} 
\label{table:models}
\end{center}
\end{table}




%poner las graficas para un modelo dado poner el efecto de
%la rotación (poner %uno con velocidad cero) y del observador. 



\section{Results}
\label{sec:results}

The central result of this paper is summarized in Figure
\ref{fig:HOMandCentral} where we show that rotation as a considerable
effect on the morphology of the emergent Lyman-alpha line both in the
case where the photons are emitted at the sphere's center and when
they are initialized with an homogeneous distribution all over the
gas volume. 

The results for this outgoing spectra are integrated over the whole
sphere, meaning that all the escaping photons were taking into account
regardless of the direction of the outgoing photons. Figure
\ref{fig:Different Observers} shows how if one gives a weight to each outgoing
photon according to its direction when escaping the gas distribution
it is posible to detect notable differences in the spectrum for
different viewing angles.



\begin{figure*}
  \includegraphics[width=0.45\textwidth]{7tDifSpeedsZ.png}
  \includegraphics[width=0.45\textwidth]{7tHOMDifSpeeds1.png}
  \label{fig:HOMandCentral}\caption{Shape of the lyman alpha line for
    differents velocities. The left (right) panel shows the central
    (homogeneous) photon distribution.}
\end{figure*}




\begin{figure}
\begin{center}
\includegraphics[scale=0.45]{Observers.png}
\end{center}
\caption{Spectra for different observers. Model:$V=300$\kms, Optical
  Depth $\tau=10^{7}$ and Central Distribution without dust.} 
  \label{fig:DifferentObservers}
\end{figure}


\subsection{Maxima Positions}
\label{sec:MP}

\begin{figure}
    \includegraphics[width=0.45\textwidth]{maximumvsODDifSpeedsCentral.png}
    \includegraphics[width=0.45\textwidth]{maximumvsODDifSpeedsHOM.png}
  \label{figure:maximavsOD}\caption{Position of the maxima in the outgoing spectra for differents Optical Depths, (up)Central Distriution, (Down) Homogeneous Distribution.}
\end{figure}


\begin{figure}
    \includegraphics[width=0.45\textwidth]{maximumvsvelocitiesDifODCentral.png}
    \includegraphics[width=0.45\textwidth]{maximumvsvelocitiesDifODHOM.png}
  \label{figure:maximumsvsvelocities}\caption{Position of the maxima in the outgoing spectra for differents Rotational velocities, (up)Central Distriution, (Down) Homogeneous Distribution.}
\end{figure}

The maximums position gives information about the wave lenght of the
mayority of the outcoming photons after they interact with the gas, in
addition as a photon have more scatterings its wave lenght would be
larger than the initial which is $1216$ {\AA}. 
As we can see in Figure ~\ref{figure:maximumsvsvelocities} the
position of the maxima does not change with rotational velocity for
the central dsitribution. On the other hand for the homogeneous
distriution the maxima position $X_{m}$ change for $Log_{10}\tau=5$,
this is because at higher rotational velocities photons escape with
less scatterings, for $v=100$\kms the majority of the photons
still escape with scatterings but for $v=200$\kms the number
of photons that escape with a few or any scatterings is higher, this
is why the maxima change to the center of the spectra.\\ 

If we now study the effect of the optical depth $\tau$ in the maxima position $X_{m}$ Figure~\ref{figure:maximavsOD}, we found that as the optical depth increase (include reference of previous articles) the maxima position increase and this is a well known result, but when rotation is included the dependence is not linear.
%(think a lit bit more on this, maybe make a regression)
%explain the method to find the maxima position
% Nscatt plots should give a hint about central maxima.
\subsection{Line Equivalent Width}
\label{sec:LW}

The width of the line provide information of
how many photons scape with a particular wave lenght, in the
ideal case in which all the photons scape with the same wave lenght
the outcoming spectrum would be narrower. \\
We found a dependency of the equivalent width with the rotation of the gas cloud for all the models we study, this is shown in Figure ~\ref{figure:width}. As velocity increase also the equivalent width increase, it means that some photons escape with fewer scatterings than the static case but at the same time some photons escape with more scatterings.\\

% I dont understand this at all, hope Nscatts plot helps 
 

\begin{figure*}
    \includegraphics[width=0.45\textwidth]{Width7Central.png}
    \includegraphics[width=0.45\textwidth]{Width7Homogeneous.png}
    \includegraphics[width=0.45\textwidth]{Width6.png}
    \includegraphics[width=0.45\textwidth]{Width6HOM.png}
    \includegraphics[width=0.45\textwidth]{Width5.png}
    \includegraphics[width=0.45\textwidth]{Width5HOM.png}
  \label{figure:width}\caption{Width of the lyman-alpha line for all the models. }
\end{figure*}

We also found that the equivalent width is not the same for differents angles of observation in particular as the angle increase the width also increase Figure ~\ref{fig:withvstheta}, this is related with the fact that photons escape easily (with less scatterings) in the $\hat{k}$ direction\\ 
Following the convention of Verhamme et al 2012 we define $\mu=cos(\theta)$, we can make a polynomial fit the $EW(\mu)$ for our models, this is resumed in Table ~\ref{table:fits}
%make comparison with Verhamme et al 

\begin{figure*}
    \includegraphics[width=0.45\textwidth]{WidthvsThetaDifODHOM.png}
    \includegraphics[width=0.45\textwidth]{WidthvsThetaDifOD.png}
    \includegraphics[width=0.45\textwidth]{WidthvsThetaDifSpeedsHOM.png}
    \includegraphics[width=0.45\textwidth]{WidthvsThetaDifSpeeds.png}
  \label{figure:widthvstheta}\caption{Width of the lyman-alpha line for all the models. }
\end{figure*}

% fix this table is too big
\begin{table*}
\begin{center}
\begin{tabular}{cc}\hline
Model & Fit\\
Central, $\tau=10^5, V=200$\kms &$EW(\mu)=-59-1677\mu-17864\mu^{2}+84509\mu^{3}-149877\mu^{4}$\\
Central, $\tau=10^6, V=200$\kms &$EW(\mu)=-1+53.8\mu-1080\mu^{2}+9644\mu^{3}-32264\mu^{4}$\\
Central, $\tau=10^7, V=200$\kms & $EW(\mu)=0.011+0.9\mu-27\mu^{2}+359\mu^{3}-1791\mu^{4}$\\
Hom, $\tau=10^5, V=200$\kms & $EW(\mu)=11.66-187.22\mu-54.58\mu^{2}+12932.5\mu^{3}-51882.9\mu^{4}$ \\
Hom, $\tau=10^6, V=200$\kms & $EW(\mu)=-361.64+17158.92\mu-305268.4\mu^{2}+2413478\mu^{3}-7154649\mu^{4}$ \\
Hom, $\tau=10^7, V=200$\kms & $EW(\mu)=-0.28+23.29\mu-721\mu^{2}+9912\mu^{3}-51074\mu^{4}$ \\
\hline
\end{tabular}
\caption{Fits for EW models} 
\label{table:fits}
\end{center}
\end{table*}




\subsection{Escape Fraction}
\label{sec:EF}

The fraction of photons that escape from the cloud of gas and dust is
defined as:\\ 

\begin{equation}
F_{e}=\dfrac{\Sigma_{NI} \vec{k}\cdot \vec{o}}{\Sigma_{NF}\vec{k}\cdot \vec{o}}
\end{equation}

Where NI is the initial number of photons and NF is the final. This
escape fraction is computed for all the models which results are
shown in Fig.[6]\\ 
 
\begin{figure*}
  \includegraphics[width=0.45\textwidth]{FECentral5.png}
  \includegraphics[width=0.45\textwidth]{FEHOM5.png}
  \includegraphics[width=0.45\textwidth]{FECentral6.png}
  \includegraphics[width=0.45\textwidth]{FEHOM6.png}
  \includegraphics[width=0.45\textwidth]{FECentral7.png}
  \includegraphics[width=0.45\textwidth]{FEHOM7.png}
  \label{figure:escape}\caption{Escape fraction for all the models. Left
    panels show the central distribution, while right panels show the
    homogeneous distribution} 
\end{figure*}

When the distribution is homogeneous the effect of velocity in the
escape fraction is clear while in the central model the effect is not
notorious. 
The main effect is that the escape fraction increase as the velocity
increase.  


\section{Discussion}
\label{sec:discussion}

\section{Observational implications}
% taking into account equivalent widths results with narrow band filters we can make a criterium in order to select wich galaxie could be seen in actual observations.

... The results derived in this paper have consequences on the
interpreation of galaxy observations in the Lyman alpha line.

\section{Conclusions}

\section*{Acknowledgements}

\section*{Appendix A: Tables}

\begin{table}
\begin{center}
\begin{tabular}{ccc}\hline          
Velocity (Km/s) & Maximum 1 position & Maximum 2 position \\ \hline
50 & -16.2695 & 16.23705 \\ 
100 & -15.66496 & 15.33504 \\ 
200 & -16.93149 & 14.56851 \\ 
300 & -13.40048 & 16.09952 \\ \hline 
\end{tabular} 
\caption{Optical Depth $\tau= 10^{7}$, Central Distribution}
\end{center}
\end{table}

\begin{table}
\begin{center}
\begin{tabular}{ccc}\hline 
Velocity (km/s) & Maximum 1 position & Maximum 2 position\\ \hline
50 & -7.46286 & 6.53714 \\ 
100 & -7.53357 & 6.96643 \\ 
200 & -8.17453 & 7.32547 \\ 
300 & -6.81487 & 6.18513 \\ \hline 
\end{tabular} 
\end{center}
\caption{Optical Depth $\tau = 10^{6}$, Central Distribution}
\end{table}

\begin{table}
\begin{center}
\begin{tabular}{ccc}\hline 
Velocity(Km/s) & Maximum 1 position & Maximum 2 position\\ \hline
50 & -4.33708 & 3.66292 \\ 
100 & -4.27326 & 3.72674 \\ 
200 & -3.7737 & 3.7263   \\ 
300 & -3.84903 & 4.15097 \\ \hline 
\end{tabular}
\caption{Optical Depth $\tau=10^{5}$, Central distribution} 
\end{center}
\end{table}

Line width\\

\begin{table}
\begin{center}
\begin{tabular}{ccc}\hline 
Velocity(Km/s) & FWHM & $\theta$\\ \hline 
50 & 12.62 & $0^{o}$ \\ 
50 & 12.72 & $45^{o}$ \\ 
50 & 12.93 & $90^{o}$ \\ 
100 & 14.07 & $0^{o}$ \\ 
100 & 14.48 & $45^{o}$ \\ 
100 & 15.00 & $90^{o}$ \\ 
200 & 16.90 & $0^{o}$  \\ 
200 & 18.51 & $45^{o}$ \\ 
200 & 21.24 & $90^{o}$ \\ 
300 & $24.69^{*}$ & $0^{o}$ \\ 
300 & $25.79^{*}$ & $45^{o}$  \\ 
300 & $27.73^{*}$ & $90^{o}$ \\ \hline
\end{tabular}
\caption{Lines Widhts for a Central Distribution and $\tau=10^{7}$} 
\end{center}
\end{table}


Escape fraction\\

\begin{table}[H]
\begin{center}
\begin{tabular}{ccccc}\hline 
Model & Velocity (km/s) & $\theta$ & Dust $\sum (s)$& $\sum (s)$\\ \hline 
Homogeneous & 50 & $0^{o}$& 13293.06 &49939.53\\ 
Homogeneous & 50 & $45^{o}$& 13291.04 &50001.59\\ 
Homogeneous & 50 & $90^{o}$ & 13348.76 &49922.73\\ 
Homogeneous & 100 & $0^{o}$ & 15527.69 &50114.11\\ 
Homogeneous & 100 & $45^{o}$ & 15511.56 &49967.17\\ 
Homogeneous & 100 & $90^{o}$ & 15401.71 & 49833.65\\ 
Homogeneous & 200 & $0^{o}$  & 17830.85 & 50078.69\\ 
Homogeneous & 200 & $45^{o}$ & 17932.87 & 50064.42\\ 
Homogeneous & 200 & $90^{o}$ & 17830.85  & 49931.748\\ 
Homogeneous & 300 & $ 0^{o}$ & 18687.33 & 50048.33 \\ 
Homogeneous & 300 & $45^{o}$ & 18572.12& 49922.67\\ 
Homogeneous & 300 & $90^{o}$ & 18421.79 & 49979.37\\ \hline
\end{tabular}
\caption{Escape fraction for a Homogeneous Distribution and optical depth $10^{5}$.} 
\end{center}
\end{table}


\begin{table}
\begin{center}
\begin{tabular}{ccccc} \hline 
Model & Velocity (km/s) & $\theta$ & Dust $\sum (s)$& $\sum (s)$ \\  \hline 
Central & 50 & $0^{o}$ & 4809.881 & 49917.069 \\ 
Central & 50 & $45^{o}$ & 4829.21 & 49811.79 \\ 
Central & 50 & $90^{o}$ & 4845.108 & 49853.039\\ 
Central & 100 & $0^{o}$ & 4809.665 & 49921.30 \\  
Central & 100 & $45^{o}$ & 4828.65 & 49820.13 \\ 
Central & 100 & $90^{o}$ & 4846.45 & 49854.0 \\ 
Central & 200 & $0^{o}$  & 4809.63 & 49917.64 \\ 
Central & 200 & $45^{o}$ & 4829.25 & 49818.49 \\ 
Central & 200 & $90^{o}$ & 4844.89 & 49856.66 \\ 
Central & 300 & $ 0^{o}$ & 4810.56 & 49922.98\\ 
Central & 300 & $45^{o}$ & 4831.16 & 49823.33\\ 
Central & 300 & $90^{o}$ & 4845.33 & 49858.48\\ \hline
\end{tabular}
\caption{Escape fraction for the central Distribution and optical depth $10^{5}$.} 
\end{center}
\end{table}


% Bibliography
%-----------------------------------------------------------------
%\begin{thebibliography}{99}

%\bibitem{Cd94} Author, \emph{Title}, Journal/Editor, (year)

%\end{thebibliography}

\end{document}
