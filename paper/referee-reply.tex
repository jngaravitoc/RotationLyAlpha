\documentclass[12pt]{article}
\usepackage[utf8]{inputenc}
\usepackage[T1]{fontenc}
\usepackage{amsmath}
\usepackage{graphicx}

\title{Response to the second report on the article draft \emph{The
    Impact of Gas Bulk Rotation on the Lyman-$\alpha$ line.}} 
\author{Juan N. Garavito-Camargo, Jaime E. Forero-Romero, Mark Disjkstra}
\date{\today}

\begin{document}


\maketitle

In what follows the comments by the referee are boldfaced.

Best regards, \\

The Authors\\


\section*{Global Comments}

\subsection*{Variations with viewing angle... again}

{\bf Figs 2 and 3 are very nice. They illustrate clearly the evolution
  of spectral shape with viewing angle. Note that the x label is
  wrong, it is not Vmax. It also needs a color bar to explicit the
  scale between a red pixel and a green/blue pixel. Because... to my
  eyes, there is an obvious third result illustrated on these Figs,
  which is that much less photons seem to escape at the equator than
  along the poles: the red pixels are concentrated towards the pole,
  whereas there seems to be less light emerging from equatorial
  directions. But if the dynamical range between red and blue is very
  small, maybe it’s ok, given that the equatorial spectrum is
  broader}

We have correctd the xlabel. Indeed the
concentration of photons in a narrow range of velocities along the
poles (i.e. red values) is compensated by a broader distribution
around the equator. The total number of photons is the same regardless
of the $\cos\theta$ value. This is the result of the test shown in
Figure 5. 




{\bf How did you select photons when you did this Fig 2 ? Did you
  select them on $|\cos\theta|$? You should select only one side, one
  hemisphere, otherwise you count twice more photons anywhere else
  than along the equator. Did you select a peculiar $\varphi$ ? Or did
  you sum over the azimuthal direction ? In order to avoid this
  impression that less photons escape at equator, you should indeed
  sum over $\varphi$. }

The 2D distribution is symmetric on $\cos\theta$. We mention this in
the first paragraph of the results section. Using this symmetry we select
by $|\cos\theta|$ in order to have less noise in the 2D histograms. We
also sum over $\varphi$.


{\bf However, these selection biases would also affect the static case, whereas the first column looks ok to me, it seems that the same “intensity” is escaping from all directions for this column, and not for the others.
It is in strong contradiction with your Fig 5. How do you explain this ?
Furthermore, I would have expected the contrary, with more photons
escaping at the equator, where the velocity field is stronger, than
through the poles, where everything is static... However, Ly$\alpha$
radiation transfer effects can often be far from intuitiv.}  

The colors in the Figures 2 and 3 correspond to number densities
inside a velocity-$|\cos\theta|$ pixel. The total flux is thus an
integrated quantity over the velocity and should not be estimated from
the pixel intensity. It is difficult to visualize
the result from Figures 2 and 3. That's why we prepare Figure 5 to
show that the integrated flux is independent of the angle. 


\subsection*{intrinsic spectrum on Fig 4}

{\bf From your Figs 6 and 7, it seems that the results for central and
  homogeneous cases are very comparable. I would like to see the shape
  and FWHM of the intrinsic spectra (I mean before transfer) in the
  case of homogeneous sources, for the static case, as the fiducial
  case, and when Vmax=100,200,300 km/s, along the equatorial
  direction. I would like to infer how much of the broadening of the
  line is really due to transfer effects. Could you make a figure
  where you compare the intrinsic and emerging spectra along the
  equatorial plane ?}  


The Figure is here:\\


\includegraphics[width=1.0\textwidth]{./intrinsic_spectrum.png}

Each column corresponds to $100$, $200$ and $300$km/s. The intrinsic
spectrum for the static case is a delta function around $x=0$.

\subsection*{Sect 4.1 : Rotation = Static ?}

{\bf I don’t understand the reasonning that rotation would act on the
  Ly$\alpha$ transfer as a static medium, since Ly$\alpha$ photons are
  “co-rotating”. You mention this argument once in the “answer to
  referee” document, and once in the text Sect. 4.1, with a sentence
  which reads absurd : a rotating sphere is identical from a static
  one. At least, it is not well formulated.

It seems to me that your argument should be right for any kind of
motion. Why would it be specific to rotation ? In other words, you
could exchange the word “rotation” with the word “outflow” or “inflow”
in your reasonning, right ? “While scattering events off atoms within
the outflowing cloud impart Doppler boosts on the Ly$\alpha$ photon,
these Doppler boost are only there in the lab-frame. Therefore, in the
frame of the outflowing gas cloud all atoms are stationnary with
respect to each other and the scattering process proceeds identical as
in the static case.”. However, all atoms in a
rotating/outflowing/infalling cloud are not stationary with respect to
each other, otherwise the cloud would be static! Your reasonning is
true at the scale of each cell, but not at global scale. 

I did not understand the alternative -more quantitative- explanation better.
Can you develop this point ? Maybe, to convince me, you could look at
the redistribution of frequencies after one single diffusion ( 1/ in
all directions, 2/ along the rotation axis, 3/ in the equator plane),
emitting your photons at a radius r = 0.9 R, and comapre them with the
same redistributions in the static case. If they are identical, you
are right, rotation does not have any effect... if they are different,
multiple scatterings will have a cumulative effect on the spectral
shape, but also on the probalility to escape more easily in some
directions than others... I understand that this implies new
simulations and more time, but I would really appreciate to see these
plots. } \\ 

These are good points. The argument that you can replace the word
'rotating' with 'outflowing' was interesting, and briefly confused
us. However, there is a clear difference between the two cases. For a
cloud undergoing solid body rotation, we can draw a line between any
two atoms within it, and their relative velocity along this line is
zero (apart from the relative velocity as a result of Thermal motion),
irrespective of the rotation velocity of the cloud. This relative
velocity is what is relevant for the RT. In contrast, if one repeats
the same exercise for the outflowing/inflowing case, then will be a
relative velocity between two atoms which depends on the
outflow/inflow velocity.\\ 

We can further illustrate this point by having one photon be emitted
in the center of the rotating cloud into the plane of the equator, and
another in the direction of the pole. Let both photons travel $\tau=1$
(at line center), and then scatter by 90 degrees, preserve its
frequency in the frame of the atom, and then again travel $tau=1$. In
both cases, the photon travels the same distance. In the LAB-frame,
both photons will have different frequencies, but not in the frame of
the gas.\\ 

The situation described above is clearly unique to solid-body
rotation. It does not apply to non-solid body rotation, which we
intend to study in upcoming work. We think our own + the referees
confusion only emphasizes that even this `simplified' problem of
solid-body rotation is highly non-trivial. We have attempted to
clarify this further in \S 3.X 

\subsection*{Angular variation of the distribution of escaping directions}

{\bf It seems to me that Ly$\alpha$ photons escape a medium through
  the path of minimum optical depth. So, in a rotating cloud, where
  the velocity field is tangential, they might “rotate” or spiral from
  the center to the edge. To test this idea, you could look at the
  distribution of escaping directions from photons emerging at the
  pole or at the equator. But for this, you need to keep track of the
  location of escape, and not only of the direction of escape. Did you
  collect this information ? 

I would expect that photons at the equator escape more tangentially to
the sphere than at the poles... But again, Ly$\alpha$ transfer is not
exactly intuitiv.} 

We have looked into this. We selected two different sets of
photons. The first one escapes through the north pole and the second
around the equator in a region with $0<\varphi<90$. For each set we
compute the dot product $\mu\equiv\hat{k}_{out}\cdot\hat{r}_{out}$,
where $\hat{r}_{out}$ is the unit vector in the radial direction for
each photon. We plot the distributions for $\mu$ for two different
$\tau_H$. In each plot we find that the two distributions (equator and
pole) are virtually the same: 
 
\begin{center}
\includegraphics[width=0.6\textwidth]{./surface_mu_tau5_diff_pos.png}
\includegraphics[width=0.6\textwidth]{./surface_mu_tau7_diff_pos.png}
\end{center}

The distribution is also insensitive to the rotational velocity:

\begin{center}
\includegraphics[width=0.6\textwidth]{./surface_mu_t5_diff_v.png}
\end{center}


The only trend can be found for different optical depths:

\begin{center}
\includegraphics[width=0.6\textwidth]{./surface_mu_diff_tau.png}
\end{center}

\section*{Details}

\subsection*{Sect. 3.1}

{\bf You say that “If the viewing angle is aligned with the rotation
  axis, $|\mu| \approx 1$, the Ly$\alpha$ line keeps in most of the
  cases a double peak”, isn’t it in all cases ?} 
  
  Yes it is doubles peaked in all the studies cases.

\subsection*{Sect. 3.5}
{\bf The parameter a is not defined when you discuss Neufeld.} 

we define the relative line width as:

\begin{equation}
a = \dfrac{\Delta \nu_{L}}{2 \Delta \nu_{D}}
\end{equation}

Where $\Delta \nu_{L}=4.03 \times 10^{-8} \nu_{0}$ is the natural line width.
and $\Delta \nu_{D} = (\nu_{p}/c)\nu_{0}$ is the Doppler frequency shift.
\subsection*{Fig9}

{\bf Is Fig 9, left panel, for a central source ? Can you show the
  homogeneous case also ? You discuss in the referee repport that the
  distribution is not bimodal anymore. But it has to be different than
  in central case, spanning the whole range of nb scatt, because
  photons are emitted at every radii in the cloud, so at every optical
  depth. On the right panel, solid and dashed lines are not defined.}
   
\begin{center}
\includegraphics[width=0.6\textwidth]{./f9h.png}
\end{center}

\subsection*{Vmax = 1000 km/s}
{\bf If you did this test and checked that the nb of scatterings and
  the escape fraction is still angle invariant for such a high
  velocity, then it is worth mentionning it. Because in principle,
  media with very high velocities should start to become transparent
  for Ly$\alpha$.}  

For solid body rotation, the results should not depend on $v_{\rm
  max}$ at all. We have even done calculations with $v_{\rm max}=10^6$
km/s, and confirmed this result. 


\end{document}
