\documentclass[12pt]{article}
\usepackage[utf8]{inputenc}
\usepackage[T1]{fontenc}
\usepackage{amsmath}
\usepackage{graphicx}

\title{Response to the second report on the article draft \emph{The
    Impact of Gas Bulk Rotation on the Lyman-$\alpha$ line.}} 
\author{Juan N. Garavito-Camargo, Jaime E. Forero-Romero, Mark Disjkstra}
\date{\today}

\begin{document}


\maketitle

We thanks the referee for the new minor comments. 
In what follows the comments by the referee are boldfaced.

Best regards, \\

The Authors\\

\section*{Minor Comments}

{\bf The color bar in Fig 2 and 3 has no units. Indeed, there is only a factor of $2$ in “flux” between a
red and a green pixel, so the broadening of the line compensates the change in color with $\theta$. I still
believe that you should comment on this in the text, because the visual impression is really that
there are less photons escaping along the equator than along the rotational axis. Especially when
you look at the top right panel.}\\

Thank you very for notest this, we put the right units and also we comment on that in the text.\\ 

{\bf could you add the intrinsic spectra for uniform sources on Fig4 ? And maybe add a fig5 which
would be the same as Fig4, but for $\theta = 0$? To illustrate the 2 extreme cases.}\\

About Fig 4. we add the intrinsic spectra, and we add an aditional figure showing the case $\theta=0$ to illustrate the 2 extremes
cases. When we add the intrinsic spectra we cut the plot to the heigh of the rotation spectra because the intrinsic spetra is to high 
and the details of the rotation spectra can not be apreciated.\\

{\bf About the directionallity of the escaping photons, why did you select only photons with $0 < \varphi < \pi/2$? Did you then normalise the blue and green curves to sample the same number of photons ? Now that I understood that Lya transfer in solid-body rotating cloud proceeds like in a static medium, until projection in an external, non-rotating frame, the only expected trend should be with optical depth, indeed. These curves show $I(\mu)$ that you discuss in the appendix, right ? It seems that the regimes that you investigated are not yet enough optically thick so that $I(\mu) = \propto \mu$. Maybe you can show these plots on the appendix, to illustrate that the analytic solution becomes better in veryoptically thick regimes.}\\

We did not limit the selection to $0 < \varphi < \pi/2$. The outgoing
photons close to the surface follow are in that range, where $\varphi$
describes the angle with respect to the normal to the surface.  The
curves have the same normalization in each of the plots.  

We have included these plots in the appendix and extended the
discussion in the direction suggested by the referee. 

\end{document}
